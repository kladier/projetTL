\hypertarget{index_intro_sec}{}\section{Introduction}\label{index_intro_sec}
En allant dans la section Classes, vous aurait accès à la documentation de l'ensemble des classes. A partir de là, vous pourrez trouvez de la doc concernant les attributs et méthodes des classes.

Dans cette page principale, nous verrons comment exécuter le logiciel Autoroute, prendre en main les sources, l'arborescence du projet ainsi que des définitions relatives à la théorie des langages et aux automates.\hypertarget{index_install_sec}{}\section{Installation du logiciel pour les utilisateurs}\label{index_install_sec}
Bla Bla Bla\hypertarget{index_dev_sec}{}\section{Pour les développeurs}\label{index_dev_sec}
\hypertarget{index_etape1}{}\subsection{Etape 1 \-: Prise en main des sources et execution}\label{index_etape1}
Ce logiciel est développé en C++, avec le framework Q\-T5. La manière la plus simple d'accéder aux sources, d'exécuter le programme et de modifier ce logiciel est la suivante \-:
\begin{DoxyItemize}
\item installer Q\-T
\item créer un dossier dans lequel vous mettrez les 3 dossiers (executable, doc et automate-\/project)
\item dans Q\-T, cliquez sur Open a project puis allez chercher le fichier automate-\/project/autoroute.\-pro
\item pour lancer le logiciel, cliquez simplement sur la flèche verte
\end{DoxyItemize}

Il vous faudra peut-\/être configurer dans l'onglet \char`\"{}\-Projects\char`\"{} l'exécution. Il suffit normalement de préciser le dossier automate/project et d'utiliser les paramètres par défaut.

N\-B \-: Vous aurez peut-\/être un problème de version si vous avez une version supérieure à Q\-T5. Il suffit en général de modifier le nom des bibliothèques. Si cela ne change pas, il vous reste plusieurs solutions \-:
\begin{DoxyItemize}
\item aller voir sur le net comment passer le projet de Q\-T5 à la version actuelle de Q\-T
\item résoudre les erreurs de compilation (aidez vous du debugger de Q\-T), c'est la solution conseillée.
\end{DoxyItemize}\hypertarget{index_etape2}{}\subsection{Etape 2 \-: Arborescence du projet}\label{index_etape2}
\begin{DoxyVerb}   - doc/ : vous trouverez ici deux dossiers (html et latex) correspondant à deux formats de la documentation. Il y aussi dans ce dossier les comptes rendus 2010 et 2015.
\end{DoxyVerb}
 Il est possible d'ouvrir ce fichier avec Doxygen et de générer la documentation du programme si vous voulez la modifier. Ce tutoriel est assez bien fait pour prendre en main doxygen \-: \href{http://franckh.developpez.com/tutoriels/outils/doxygen/}{\tt http\-://franckh.\-developpez.\-com/tutoriels/outils/doxygen/}

\begin{DoxyVerb}   - automate-project/ : les sources du programme.
\end{DoxyVerb}
 Mieux vaux ne pas y toucher au début, surtout si l'on ne connait pas Q\-T et modifier le code seulement via Q\-T. \begin{DoxyVerb}  - executable/ : tout les fichiers relatifs aux exécutables
\end{DoxyVerb}
\hypertarget{index_definitions}{}\section{Définitions}\label{index_definitions}
\hypertarget{index_minimisation}{}\subsection{Minimisation d'un automate}\label{index_minimisation}
\hypertarget{index_standardisation}{}\subsection{Standardisation d'un automate}\label{index_standardisation}
\hypertarget{index_produit}{}\subsection{Produit de deux automates}\label{index_produit}
\hypertarget{index_determinisation}{}\subsection{Déterminisation de deux automates}\label{index_determinisation}
